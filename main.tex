%%%%%%%%%%%%
%% This is the CWRU thesis template for PhD and Masters degree theses. The template file cwru_thesis.cls and this main.tex file follow the guidelines of the school of graduate studies found at https://case.edu/gradstudies/current-students/electronic-theses-and-dissertation-guidelines as of February 2023, with the inclusion of an additional List of Symbols if desired.
%%
%% This .tex file is for use with BibLaTeX.
%%
%% As of Spring 2023, the School of Graduate Studies requires some minimum accessibility requirements for all electronic theses. Accessibility is difficult to produce with LaTeX on the front end, but fortunately it is quite easy to meet the requirements on the back end using Adobe (not Adobe Reader). Simply download the PDF of your thesis and follow this guide: https://case.edu/gradstudies/sites/case.edu.gradstudies/files/2023-02/CWRU%20Thesis%20and%20Dissertation%20Accessiblity%20Guide_0.pdf
%%%%%%%%%%%%

\documentclass[12pt]{cwru_thesis}
\usepackage[hyphens]{url}
\usepackage{lipsum}
\usepackage{graphicx}
\usepackage{hyperref}
\hypersetup{colorlinks=true, linkcolor=black, filecolor=black, urlcolor=black, citecolor=black}
\urlstyle{same}
\usepackage[acronym,toc]{glossaries}
\usepackage[intoc]{nomencl}
\renewcommand{\nomname}{List of Symbols}

\usepackage{mathptmx}
\usepackage[T1]{fontenc}
\usepackage[utf8]{inputenc}
\usepackage{amsfonts, amsmath, amsthm, amssymb}


\newcommand{\dif}{\mathrm{d}}
\newcommand{\Dif}{\mathrm{D}}
\renewcommand{\vec}[1]{\ensuremath{\underline{#1}}}
\newcommand{\grad}{\underline{\nabla}}
\newcommand{\curl}{\underline{\nabla} \times}
\newcommand{\lap}{\underline{\nabla}^2}
\renewcommand{\div}{\underline{\nabla} \cdot}
\newcommand{\paren}[1]{\left( {#1} \right)}
\newcommand{\norm}[1]{\lVert#1\rVert}

% Must use biblatex to produce the Published Contents and Contributions, per-chapter bibliography (if desired), etc.
\usepackage[
    backend=biber,natbib,style=numeric-comp,maxbibnames=99
]{biblatex}

% Name of your .bib file(s)
\addbibresource{example.bib}

\makeglossaries
%%%%%%%%%%%%%%%%%%%%%%%%%%%%%%%%%%%%%%%%%%%%%%%%%%%%%%%%%%%%%%%%%%%%%%%%%%%%%%%%%%%%%%%
%Glossary entries


%Acronyms to include in the list of acronyms
\newacronym{leo}{LEO}{Low Earth Orbit}


\begin{document}
\pagenumbering{roman}

% Do remember to remove the square brackets!
\title{Omar's Thesis} %Title of your thesis in title case
\author{Omar Loudghiri} %Your name

\degreeaward{Master's of Science in Computer Science}                 % Degree to be awarded
\department{Department of Computer Science and Data Science}
\university{Case Western Reserve University}    % Institution name  
\unilogo{cwru_logo.eps}                                 % Institution logo
\defendmonth{August, 2024}          % Graduation month and year
\defenddate{July 10th, 2024}          % Date of thesis defense

%Committee Member names. If you have a different number of committee members for your defense, you will need to edit lines 183-190 of cwru_thesis.cls accordingly.
\committeeChair{An Wang} %Committee Chair's name
\committeeOne{Mark Allman} %Committee member #1's name
\committeeTwo{Vincenzo Liberatore} %Committee member #2's name
\committeeThree{Mehmet Koyuturk} %Committee member #3's name

%%  If you'd like to add the CWRU logo from your title page, simply add the "[logo]" text to the maketitle command. Note that the School of Graduate Studies doesn't like this.
\maketitle

\begin{dedication} 	 
  TBD
\end{dedication}

\begin{KeepFromToc}
  \tableofcontents
\end{KeepFromToc}
\listoftables
\listoffigures



\begin{acknowledgements}
   TBD
\end{acknowledgements}

%If you have acroynms you wish to define, include this line. If not, you may delete it. See https://www.overleaf.com/learn/latex/Glossaries#Acronyms for more information about how to use Acronyms.
\printglossary[type=\acronymtype, title=List of Abbreviations]

%If you have terms you wish to define in a glossary, include this line. If not, you may delete it. See https://www.overleaf.com/learn/latex/Glossaries for more information about how to use Glossaries
\printglossary

%If you have a nomenclature section for defining symbols, include this line. If not, you may delete it. See https://www.overleaf.com/learn/latex/Nomenclatures for more information about how to use Nomenclatures
\printnomenclature

\begin{abstract}
  TBD
\end{abstract}

\mainmatter

\setcounter{secnumdepth}{2}

\chapter{Introduction} \label{chap:intro}
\section{Load Balancing} \label{sec:Loadsection}
Load balancing is a key network management technique that distributes traffic across multiple servers, preventing any single server from becoming overwhelmed. Originally developed as network-based hardware, load balancing now plays a crucial role in modern infrastructure by ensuring high availability, scalability, security, and performance.

Applications today must handle millions of simultaneous sessions. Load balancers dynamically distribute traffic across servers with duplicate data, ensuring reliable and fast data delivery. This process also provides redundancy; if a server fails, traffic is redirected to maintain continuous access.

Load balancing enhances security by minimizing attack surfaces and rerouting traffic if a server is compromised. 

It also optimizes performance by managing resource use and traffic spikes. Various algorithms, such as round-robin and least connections, help distribute traffic based on real-time conditions.


The goal of this project is to measure load balancing behavior on the Internet and map the presence of load balancers worldwide. This will enhance our understanding of their impact on network performance and security.


\section{Areas of Study} \label{sec:Areassection}





% \subsection{This is a Subsection} \label{subsec:exsubsec}







\chapter{Related Works} \label{chap:intro}
\section{Paris Traceroute}

\subsection{General Introduction}
The traditional model of the Internet assumes a single path between a pair of end-hosts. However, modern commercial routers often include load balancing capabilities, creating multiple active paths between hosts. This shift challenges the conventional single-path assumption used by many Internet applications, network simulation models, and measurement tools.

In their paper \textit{Measuring Load-balanced Paths in the Internet}, Brice Augustin, Timur Friedman, and Renata Teixeira from LIP6 at Université Pierre et Marie Curie present a comprehensive study on load-balanced paths. They highlight the significance of recognizing load balancing in contemporary networking by demonstrating how it affects traffic distribution and path diversity.

The authors enhance a traceroute-like tool called \textit{Paris traceroute}, designed to find all paths between a pair of hosts. Their methodology involves identifying load-balancing routers and characterizing the load-balanced paths. By conducting measurements from 15 sources to over 68,000 destinations, their study reveals that the traditional single-path concept no longer holds. They found that 39\% of source-destination pairs traverse a load balancer, and this percentage rises to 70\% when considering paths between a source and a destination network.

Their work contributes significantly to understanding Internet path diversity by:
\begin{enumerate}
    \item Developing Paris traceroute’s Multipath Detection Algorithm (MDA) to find all paths from a source to a destination under different types of load balancing.
    \item Characterizing the load-balanced paths in terms of length, width, and asymmetry.
    \item Establishing a methodology to measure round-trip times (RTTs) of load-balanced paths, considering delays on both forward and return paths.
\end{enumerate}

This study underscores the necessity for the research community to reconsider the concept of a single Internet path and highlights the importance of accurately measuring and understanding load-balanced paths. The insights gained from this work are critical for developing more realistic network models and improving the design and reliability of Internet applications.

\subsection{Multipath Detection Algorithm (MDA)}

The Multipath Detection Algorithm (MDA) is a key component of Paris traceroute, designed to identify and trace multiple load-balanced paths between a source and a destination. Traditional traceroute tools often fail to detect load balancing because they assume a single path. In contrast, MDA systematically discovers all paths by varying flow identifiers in probe packets.

The MDA operates hop-by-hop, sending probes to identify all interfaces at each hop. For a given interface \(r\) at hop \(h-1\), MDA generates several flow identifiers to ensure probes reach \(r\). It then sends these probes one hop further to discover the next-hop interfaces \(s_1, s_2, \ldots, s_n\).

To determine the number of probes \(k\) needed to discover all paths with a high degree of confidence, MDA assumes \(r\) is part of a load balancer that splits traffic evenly across \(n\) paths. If fewer than \(n\) interfaces are found, MDA stops. Otherwise, it increases \(n\) and sends additional probes to test the hypothesis.

To identify whether a load balancer uses per-packet or per-flow balancing, MDA sends probes with a constant flow identifier. If responses come from multiple interfaces, it indicates per-packet balancing. If all responses come from the same interface, it suggests per-flow balancing. MDA uses statistical methods to ensure a high level of confidence (typically 95%) in its classification.

For instance, to reject the hypothesis of \(n = 2\) with 95\% confidence, MDA sends \(k = 6\) probes. If load balancing across up to 16 interfaces is suspected, MDA may send up to \(k = 96\) probes to ensure all paths are discovered. This process allows MDA to effectively enumerate all paths and classify the type of load balancing in use.




\subsection{Usage of Paris Traceroute}
Our measurement methodology is deeply influenced by the work of Augustin et al. We employ Paris traceroute and its Multipath Detection Algorithm to identify and characterize load-balanced paths in our study. The detailed usage and application of these tools in our research will be discussed in a later section, where we outline our experimental setup and data collection techniques.


\chapter{This is the Second Chapter}


\chapter{This is the Third Chapter}
See? I can reference the first chapter from all the way down here. I think the first chapter was Chapter

\chapter{This is the Fourth Chapter}
\chapter{This is the Fifth Chapter}
\chapter{This is the Sixth Chapter}
\chapter{This is the Seventh Chapter}
\chapter{This is the Eighth Chapter}


%If you have appendices to your thesis, place them here
\appendix

\chapter{Questionnaire}

\printbibliography[heading=bibintoc]

\end{document}
